\documentclass[10pt,a4paper]{moderncv}

\moderncvtheme[blue]{ysc} 
\usepackage[utf8]{inputenc}
\usepackage[T1]{fontenc}
\usepackage{lmodern}
\usepackage[top=1cm, bottom=2cm, left=1cm, right=1cm]{geometry}
\usepackage{multirow}%,tabularx}

\usepackage{enumitem}
\setitemize{leftmargin=20pt}
\setlist[itemize]{itemsep=0pt, topsep=0pt}

\newcommand{\entrysep}{%
   \vspace{0.75em}
}

% ------------------------------------------------------------------------------
%      HEADER
% ------------------------------------------------------------------------------

\firstname{Yankel}
\familyname{Scialom}
\title{Ingénieur Systèmes Embarqués}
\email{yankel.scialom@mail.com}
\extrainfo{%
       \href{http://github.com/yscialom}{github.com/yscialom}\makenewline
       \href{http://fr.linkedin.com/pub/yankel-scialom/85/596/8b7/}{profil Linkedin}\makenewline
       27 ans --- permis B}
\photo[64pt]{ysc} 
\quote{Jeune ingénieur passionné, cherche emploi R\&{}D systèmes embarqués}
\phone{+336 87 63 83 13}

\AtBeginDocument{
    \hypersetup{
       pdftitle      = {Yankel Scialom -- Ingénieur Systèmes Embarqués - Linux/Android -- CV},
       pdfsubject    = {CV de Yankel Scialom},
       pdfkeywords   = {Yankel Scialom, CV, curriculum vitæ, CV, linux, embarqué, ingénieur systèmes embarqués, C, bash, shell, intégration, android}}
}

\nopagenumbers{}                         
\begin{document}
\maketitle

% ------------------------------------------------------------------------------
%      COMPÉTENCES
% ------------------------------------------------------------------------------

\section{Compétences}
\cvlineWithMargin{Développement embarqué}{%
       C/C++, ASM, VHDL, Java pour Android, noyau linux, fonctionnement d'un ALU, ...
}
\cvlineWithMargin{Langages de développement}{%
       C/C++ (Qt, Boost), Java EE (Servlet, EJB,   JDBC, ...), Web (Linux, Apache,
       [My|Postgre]SQL, PHP, HTML, CSS, Javascript, jQuery[-ui], Node.js ...).
}
\cvlineWithMargin{Divers outils et langages généraux}{%
       OCaml, Lisp, ActionScript, VirtualBasic, python, Delphi, bash, réseaux,
       administration système, ...
}
\cvlineWithMargin{Protocoles}{%
       CAN, UART, I2C, SPI, USB, GPS, TCP/IP, FTP, HTTP, DNS ...
}
\cvlineWithMargin{Outils}{%
       Git, SVN, ClearCase, Trac, suivis de bugs --- méthodes agiles et cycle en V.
}
\cvlineWithMargin{Divers}{%
       Licences Open-Source --- Lecture de \textit{datasheets} --- schémas
       électroniques --- Conception de documentation~--- Bon relationnel.
}

% ------------------------------------------------------------------------------
%      EXPÉRIENCES
% ------------------------------------------------------------------------------

\section{Expériences professionnelles}
%%
% Cap FI
\cventry{Juillet 2014--présent}{
Ingénieur conception logicielle chez \textsc{Tikehau} Investment Managment}
{~Cap Fi Technologie}{capfi}{Paris}{France}{%
   \vspace{0.75em}
   \begin{cvcolumns}
      \cvcolumn{Développeur d'outils informatiques~: C\#, Sophis toolkit, Oracle}{%
         \begin{itemize}
            \item Analyse des besoins auprès des gestionaires de portefeuille~;
            \item conception d'outils d'aide au trading, de contrôle des risques, etc.~;
            \item développement, test et mise en production.
         \end{itemize}
      }
      \cvcolumn{Développeur de routines de communication~: Batch, XSL, FTP}{%
         \begin{itemize}
            \item Actualisation des outils de communication avec les sources de prix~:
            \begin{itemize}
               \item Bloomberg,
               \item Reuters,
               \item etc.~;
            \end{itemize}
            \item Actualisation des outils de communication avec le dépositaire de \textsc{Tikehau} IM.
         \end{itemize}
      }
   \end{cvcolumns}
}
\entrysep

%%
% Astek
\cventry{Janvier--Juillet 2014}{
Ingénieur conception logicielle chez \textsc{Thales Raytheon} Systems}
{~Astek Industrie}{astek}{Massy}{France}{%
   \vspace{0.75em}
   \begin{cvcolumns}
      \cvcolumn{Architecte et développeur logiciel C++/Java}{%
         \begin{itemize}
            \item Conception, développement et intégration de fonctionnalités IHM et communication~;
            \item mise en place de procédures de tests multi-plateforme~;
            \item suivi de fait technique.
         \end{itemize}
      }
      \cvcolumn{Manager de centre de service}{%
         \begin{itemize}
            \item Assistant au directeur de projet~;
            \item gestion technique des devis \&{} facturations client.
         \end{itemize}
      }
   \end{cvcolumns}
}
\entrysep

%%
% Sysnav
\cventry{2012}{Ingénieur R\&{}D système embarqué}{Sysnav}{sysnav}{Vernon}{France}{%
   \begin{cvcolumns}
      \cvcolumn{Conception \&{} développement d'une HAL}{%
         \begin{itemize}
            \item Configuration du micro-contrôleur ;
            \item protocoles de communication (UART, SPI, USB) ;
            \item utilisation transparente du DMA, de RAM externe/interne, de la configuration de l'ALU ...
         \end{itemize}
      }
      \cvcolumn{Conception d'un système d'exploitation temps-réel}{%
         \begin{itemize}
            \item Ordonnanceur ;
            \item émulation de mémoire statique ;
            \item virtualisation des E/S pour le programme embarqué.
         \end{itemize}
      }
   \end{cvcolumns}
%
   \begin{enumerate}[label=]
      \item \textbf{Portage de l'API sur desktop Windows et Linux pour simulation \&{} debug.}
      \item \textbf{Documentation complète de l'API.}
   \end{enumerate}
}
\entrysep

%%
% TOSA
\cventry{2010}{Développeur junior}{\textsc{Thales Optronique} S.A.}{thales}{Élancourt}{France}{%
Développement d’une interface homme-machine pour une suite logicielle de
gestion de drones aériens.
}
\entrysep

%%
% Projets
\cventry{2013}{}{}{}{}{}{%
       \begin{cvcolumns}
              \cvcolumn{Projets personnels}{%
                     Logiciel opensource \href   {https://github.com/yoannsculo/JobCatcher}{JobCatcher},
                     \textit{Pentest box} Android, serveur SMS embarqué, agent de notification
                     pour \textit{desktop} gnome.
              }
              \cvcolumn{Autoformation}{%
                     \textit{Online Standford Continuing Studies (physics)},
                     sécurité informatique et \textit{pentests},
                     mathématiques pour l'ingénieur,
                     protocoles réseau, administration système~(linux),
                     divers.
              }
              \cvcolumn{Voyages}{%
                     Londres, Oxford, Canterbury (RU) ; Bavière (Allemagne) ; Belgrade, Ra\v{c}a (Serbie)~;
                     France (nombreuses villes) ; Barcelone (Espagne) ; Vietnam (circuit touristique).
              }
       \end{cvcolumns}
}

% ------------------------------------------------------------------------------
%      FORMATION
% ------------------------------------------------------------------------------
\section{Formation}
\cventry{2005--2012}{%
       Diplôme d’ingénieur en Systèmes d’Information et Télécommunications
}{Université de technologie de Troyes (UTT)}{utt}{Troyes}{France}{%
       Specialité technologies mobiles et systèmes embarqués.\newline
       Validation de la première année en systèmes mécaniques en 2009.
}

% ------------------------------------------------------------------------------
%      LANGUES
% ------------------------------------------------------------------------------
\section{Langues}
\cvlanguage{Français}{Langue maternelle}{}
\cvlanguage{Anglais}{Courant}{BULATS : B2 en 2010, amélioré depuis.}
\cvlanguage{Allemand}{Notions}{Bon niveau scolaire en 2006.}
\cvlanguage{Serbo-croate}{Apprentissage}{Méthode ASSIMIL.
}

% ------------------------------------------------------------------------------
%      AUTRES
% ------------------------------------------------------------------------------
\section{Autres}
\cvlineWithMargin{Associatif}{%
       Président de clubs étudiants de photographie et de jeux de société.
}
\cvlineWithMargin{Loisirs}{%
       Cinéphile, photographe amateur, joueur de volley ball, lecteur de bande
       dessinée, adèpte du \textit{Do It Yourself}, passionné de technologies,
       abonné à la presse scientifique.
}
\end{document}
